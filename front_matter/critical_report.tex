\documentclass[parskip=full]{scrreprt}

\RedeclareSectionCommand[pagestyle=plain,afterskip=10pt plus 1pt]{chapter}
\setkomafont{chapter}{\mdseries\headingfont\fontsize{40}{40}\selectfont\color{black!80}}
\setkomafont{pageheadfoot}{\normalsize}

\def\pnumbox#1{#1\hspace*{8cm}}
\def\gobble#1{}
\DeclareTOCStyleEntry[
  indent=0pt,
  beforeskip=0pt,
  entryformat=\itshape,
  entrynumberformat=\textcolor{oldred},
  numwidth=2em,
  linefill=\hfill,
  pagenumberbox=\pnumbox,
  pagenumberformat=\itshape
]{tocline}{part}

% use commented values if there are no parts
\DeclareTOCStyleEntry[
  indent=0pt,
  beforeskip=0pt,
  entrynumberformat=\textcolor{oldred},
  numwidth=2em,
  linefill=\hfill,
  pagenumberbox=\pnumbox
]{tocline}{section}

\DeclareTOCStyleEntry[
  indent=0pt,
  beforeskip=-\parskip,
  entrynumberformat=\gobble,
  entryformat=\ltseries,
  numwidth=2em,
  linefill=\hfill,
  pagenumberbox=\pnumbox,
  pagenumberformat=\ltseries
]{tocline}{subsection}

\usepackage{polyglossia}
\setdefaultlanguage{english}

\usepackage{fontspec}

\setmainfont{Source Sans Pro}[
  ItalicFont = Source Sans Pro Italic,
  BoldFont = Source Sans Pro Bold,
  BoldItalicFont = Source Sans Pro Bold Italic,
  FontFace = {lt}{n}{Source Sans Pro Light},
  FontFace = {lt}{it}{Source Sans Pro Light Italic},
  FontFace = {sb}{n}{Source Sans Pro Semibold},
  FontFace = {sb}{it}{Source Sans Pro Semibold Italic},
  Numbers = Proportional,
  Ligatures = Common
]
\DeclareRobustCommand{\ltseries}{\fontseries{lt}\selectfont}
\DeclareRobustCommand{\sbseries}{\fontseries{sb}\selectfont}
\DeclareTextFontCommand{\textlt}{\ltseries}
\DeclareTextFontCommand{\textsb}{\sbseries}

\newfontfamily\headingfont{Fredericka the Great}

\usepackage[pass]{geometry}
\newgeometry{twoside,inner=20mm,outer=40mm,top=20mm,bottom=40mm}

\usepackage{url}
\urlstyle{same}

\usepackage{microtype}
\microtypesetup{verbose=silent}

\usepackage{booktabs,array,longtable}
\newcolumntype{L}[1]{>{\raggedright\let\newline\\\arraybackslash\hspace{0pt}}p{#1}}

\usepackage{graphicx}

\usepackage{xcolor}
\definecolor{oldred}{rgb}{.8313,0,0}

\usepackage{pdfpages}

\usepackage{scrlayer-scrpage}
\pagestyle{scrheadings}
\clearscrheadfoot
\cfoot[\thepage]{\thepage}
\pagenumbering{roman}

\usepackage{enumitem}
\setlist[description]{%
	labelindent=2em,%
	labelwidth=6.5em,%
	leftmargin=8.5em,%
	labelsep=0pt,
	first=\ltseries,%
	font=\normalfont\itshape\ltseries%
}
\def\lyrefitem#1{\item[\lyref{#1}]}


\makeatletter

\def\@firstofthree#1#2#3{#1}
\def\@secondofthree#1#2#3{#2}
\def\@thirdofthree#1#2#3{#3}
\def\@firstoffour#1#2#3#4{#1}
\def\@secondoffour#1#2#3#4{#2}
\def\@thirdoffour#1#2#3#4{#3}
\def\@fourthoffour#1#2#3#4{#4}
\def\Dotfill{\leavevmode\cleaders\hb@xt@ .75em{\hss .\hss }\hfill \kern \z@}

\def\lyrefnumber#1{\expandafter\@setref\csname r@#1\endcsname\@firstofthree{#1}}
\def\lyreftitle#1{\expandafter\@setref\csname r@#1\endcsname\@secondofthree{#1}}
\def\lyrefpage#1{\expandafter\@setref\csname r@#1\endcsname\@thirdofthree{#1}}

\def\lyrefgenrenumber#1{\expandafter\@setref\csname r@#1\endcsname\@firstoffour{#1}}
\def\lyrefgenregenre#1{\expandafter\@setref\csname r@#1\endcsname\@secondoffour{#1}}
\def\lyrefgenretitle#1{\expandafter\@setref\csname r@#1\endcsname\@thirdoffour{#1}}
\def\lyrefgenrepage#1{\expandafter\@setref\csname r@#1\endcsname\@fourthoffour{#1}}

\def\lyrefpart#1{%
	\begingroup%
	\makebox[0pt][l]{\sbseries\color{oldred}\lyrefnumber{#1}}\hspace*{2em}%
	\makebox[0pt][l]{\sbseries\lyreftitle{#1}}\hspace*{6.5em}%
	\hfill\sbseries\lyrefpage{#1}%
	\endgroup%
}
\def\lyrefsection#1{%
	\begingroup%
	\makebox[0pt][l]{\color{oldred}\lyrefgenrenumber{#1}}\hspace*{2em}%
	\makebox[0pt][l]{\ltseries\lyrefgenregenre{#1}}\hspace*{6.5em}%
	\lyrefgenretitle{#1}\Dotfill\lyrefgenrepage{#1}%
	\endgroup%
}
\InputIfFileExists{../out/lilypond.ref}{}{\InputIfFileExists{../lilypond.ref}{}{}}


\newcommand\fancytitlehead{
	\headingfont%
	\fontsize{80}{80}\selectfont\textcolor{black!80}{\@ifundefined{@shortname}{\@lastname}{\@shortname}.}\\[15pt]%
	\fontsize{60}{60}\selectfont\@ifundefined{@shorttitle}{\@title}{\@shorttitle}.%
}


\def\firstname#1{\def\@firstname{#1}}
\def\lastname#1{\def\@lastname{#1}}
\def\shortname#1{\def\@shortname{#1}}
\def\shorttitle#1{\def\@shorttitle{#1}}
\def\namesuffix#1{\def\@namesuffix{#1}}
\def\instrumentation#1{\def\@instrumentation{#1}}
\def\parts#1{\def\@parts{#1}}

\firstname{\relax}
\lastname{\relax}
\namesuffix{\relax}
\instrumentation{\relax}
\parts{\relax}


\def\maketitle{%
\begin{titlepage}%
	\Large%
	{\@titlehead}%
	\vfill%
	{\strut\@firstname}\\%
	{\sbseries\color{oldred}\strut\@lastname}\\%
	{\strut\@namesuffix}%
	\vfill%
	{\sbseries\@title}\\%
	{\@subtitle}\\[\baselineskip]%
	{\itshape\@instrumentation}%
	\vfill%
	{\itshape\@parts}\hspace*{\fill}\raisebox{0pt}[0pt][0pt]{\includegraphics{ees_logo}}%
\end{titlepage}%
}


\newif\iftemplate\templatetrue
\newif\ifprintreport\printreportfalse
\directlua{
scores = {
  fl1 = "Flauto I",
  fl2 = "Flauto II",
  ob1 = "Oboe I",
  ob2 = "Oboe II",
  fag1 = "Fagotto I",
  fag2 = "Fagotto II",
  cor12 = "Corno I, II in?",
  vl1 = "Violino I",
  vl2 = "Violino II",
  vla = "Viola",
  soli = "Soli",
  coro = "Coro",
  org = "Organo",
  b = "Bassi",
  full_score = "Full Score"
}

last_arg = arg[\string#arg]
texio.write("Last argument: " .. last_arg)
if not (scores[last_arg] == nil) then
  tex.print("\string\\def\string\\lypdfname{" .. last_arg .. "}")
  tex.print("\string\\parts{" .. scores[last_arg] .. "}")
  if (last_arg == "full_score") then
    tex.print("\string\\printreporttrue")
  end
end
}


\@ifundefined{lypdfname}{%
  \templatefalse
  \printreporttrue
  \parts{Draft}
}{\templatetrue}

\makeatother






\begin{document}
\frenchspacing

\titlehead{\fancytitlehead}
\firstname{Johann Ernst}
\lastname{Eberlin}
\title{Die gläubige Seele}
\subtitle{Oratorium\\(F-P MS-2008)}
\instrumentation{S, A, T, B (solo), S, A, T, B (coro),\\2 fl, 2 ob, 2 fag, 2 cor, 2 trb, 2 vl, vla, b, org}
\maketitle


\thispagestyle{empty}

\vspace*{\fill}

\raisebox{-4mm}{\includegraphics{byncsaeu}}\hspace*{1em}Wolfgang Esser-Skala, 2020

© 2020 by Wolfgang Esser-Skala. This edition is licensed under the Creative Commons Attribution-NonCommercial-ShareAlike 4.0 International License. To view a copy of this license, visit \url{http://creativecommons.org/licenses/by-nc-sa/4.0/}.

Music engraving by LilyPond 2.18.0 (\url{http://www.lilypond.org}).\\
Front matter typeset with Source Sans Pro and Fredericka the Great.

\textit{First version, December 2020}

\vspace*{2cm}

\ifprintreport
\chapter*{Critical Report.}

This edition bases upon the autograph manuscript in the Bibliothèque nationale de France, Département de la Musique. The digital version of the manuscript is available at \url{https://gallica.bnf.fr/ark:/12148/btv1b10537269n} (siglum MS-2008).

In general, this edition closely follows the manuscript. Any changes that were introduced by the editor are indicated by italic type (lyrics, dynamics and directions), parentheses (expressive marks and bass figures) or dashes (slurs and ties). Accidentals are used according to modern conventions. Asterisks denote changes that are clarified in the detailed remarks below.\footnote{Abbreviations: A, alto; B, bass; b, basses; cor, horn; fag, bassoon; fl, flute; Ms, manuscript; ob, oboe; org, organ; r,~rest; S, soprano; T, tenor; timp, timpani; trb, trombone; vl, violin; vla, viola.}

\bigskip

\begin{longtable}{lll L{10cm}}
	\toprule
	\itshape Mov. & \itshape Bar & \itshape Staff & \itshape Note \\
	\midrule \endhead
	1 & 27 & vl 2 & last eighth in Ms: b8 \\
	7 & 19 & vla  & 2nd eighth in Ms: fis′8 \\
	\bottomrule
\end{longtable}

%Bass figures appear in the following movements (bars in parentheses):. The remaining bass figures were added by the editor.

Maria, die Mutter Jeſus (A)
Evangelium (T)
DIe gläubige Seele (T)
Tochter Zion (S)

This edition has been compiled and checked with utmost diligence. Nevertheless, errors and mistakes cannot be totally excluded. Please report any errors and mistakes to \url{wolfgang@esser-skala.at} or create an issue or pull request on the edition’s GitHub page \url{https://github.com/skafdasschaf/eberlin-die-glaeubige-seele}. Your help will be greatly appreciated.

\bigskip
\textit{Salzburg, December 2020\\
Wolfgang Esser-Skala}

\cleardoublepage
\chapter*{Contents.}



\lyrefsection{iztkomm}

\begin{description}
	\item[Soprano]
	Izt komm, izt komm, o rauher Sünder,\\
	kömmt, kömmt, ihr Menſchenkinder,\\
	kömmt, ſeht den Kreuzweeg an,\\
	ja ſeht was Gott vor euch gethan.\\
	Komm, Sünder! ſieh des Heilands Plagen,\\
	komm, Sünder, hilf ihm den Kreuzpflock tragen.\\
	Die Laſt, die ſeine Schulter drückt,\\
	hat deine Sünd ihm zugeſchickt.
\end{description}

\lyrefsection{meinsohn}

\begin{description}
	\item[Maria, die Mutter Jeſus]
	Mein Sohn! o Gottes Sohn!\\
	So geheſt du dann izt zum Tod?\\
	O harter Seelenſchmerz, o Pein! o Noth!\\
	(mir poltert faſt das Herz)\\
	Was kann wohl ich gekränkte Mutter ſagen?\\
	Ach ja! mein Sohn, den ich neun Monath lang\\
	in meinem Leibe truge,\\
	den dieſe Mutterbruſt mit ihrer Milch geſäuget –\\
	o Freud! o Troſt! o meine Luſt!\\
	du wirſt durch dieſen Kreuzespflock\\
	faſt biß zur Erd gebeuget:\\
	Man reißt dich ſchon zu deiner Pein,\\
	drum dringet ſich in meine Seele\\
	das Schwert des Schmerzens ein.\\
	So ſtärke dann, mein Gott,\\
	ach ſtärke doch den ſchwachen Geiſt,\\
	denn ſonſten deine Marter\\
	(was ich für dich gern wollt)\\
	mit dir, mein Sohn! gar ſterben heißt.
\end{description}

\lyrefsection{weintmit}

\begin{description}
	\item[Maria, die Mutter Jeſus]
	Weint mit mir, ihr frommen Seelen,\\
	euer Vater ſtirbt, mein Kind.\\
	Willſt, o Menſch, die Wunden zählen,\\
	ach, ſo zähle deine Sünd.\\
	Dieſe war das Maaß der Schmerzen,\\
	die ſein mater Leib erträgt,\\
	dieſe war, die ſeinem Herzen\\
	ſo viel Marter aufgelegt.
\end{description}

\lyrefsection{sobalddie}

\begin{description}
	\item[Evangelium]
	Sobald die Juden nun auf Golgatha gekommen,\\
	ſo wurden ihm die Kleider weckgenommen.\\
	Drauf hat man ihn mit Wein und Gall getränkt\\
	und an das Kreuz gehenkt.
	
	\item[Die gläubige Seele]
	Ach Anblick! der mein Herz zerbricht.\\
	Wie wird mein Heiland von dieſen Schärgenknechten\\
	ſo ſchändlich zugericht.\\
	Die ganze Höllenwuth iſt in der Judenbruſt beysamen.\\
	O Schrecken! ach! o Herzensgrauen!\\
	Hier reißen ihre Fingerklauen\\
	das milde Lamm mit vollem Grimm zu Boden.\\
	Sie fangen an, die Nerven und die Sehnen\\
	ihm ganz erbärmlich auszudehnen,\\
	die zarten Vaterhände\\
	mit größtem Ungeſtüm zu ſtrecken.\\
	O Jeſus! ach dein Leiden muß in mir\\
	das ſtrengſte Leid erwecken.
\end{description}

\lyrefsection{moerderdazen}

\begin{description}
	\item[Die gläubige Seele]
	Mörderdazen, Mörderklauen,\\
	läßt mich nicht mehr Laſter ſchauen,\\
	denkt, was Eure Boſheit thut!\\
	Dürſtet euch nach Menſchenblut,\\
	ſo nihm meins, du Henkersbrut.\\
	Ich bin, ſo die Schuld begangen,\\
	drum läßt mich den Tod erlangen,\\
	Jeſus iſt von Sünden rein,\\
	ich, ich muß beſtrafet ſeyn.
\end{description}

\lyrefsection{kaumals}

\begin{description}
	\item[Evangelium]
	Kaum als das Kreuzigen vorüber war,\\
	ſo warf die Henker Schaar\\
	das Looß um ſein Gewand,\\
	und über ſeinem Haubte ſtand\\
	der Titel ſo geſchrieben:\\
	Hier hänget Jeſus der Nazarener,\\
	der Juden König.\\
	Drauf haben alle, die vorbey gegangen,\\
	zu ſpotten angefangen,\\
	und auch die Zwey,\\
	ſo gleich zur Seite hangen,\\
	die riefen ohne Scheu:
	
	\item[Chor]
	Ey ſeht den König an!\\
	Du biſt ja Gottes Sohn,\\
	ſo ſteig vom Keuz herab.\\
	Ach ſeht, wie er ihm helfen kann.
	
	\item[Evangelium]
	Und eine düſtre Finſterniß entſtand\\
	hier in der ſechſten Stunde\\
	im ſelben ganzen Land.\\
	Man ſah die Sonn erblaßen\\
	biß auf die neunte Stunde.\\
	Da riefe Jeſus laut:\\
	Mein Gott, mein Gott,\\
	wie haſt du mich verlaßen?\\
	Dann ſaget er: Mich dürſtet.
\end{description}

\lyrefsection{wiederhirsch}

\begin{description}
	\item[Tochter Zion]
	Wie der Hirſch ganz ſchnelle\\
	zu der Waßerquelle,\\
	die ihm ſein Schmachten heilt,\\
	mit großen/frohen Schritten eilt,\\
	ſo dürſtet dich nach meiner Seele.\\
	Doch du haſt ſie ſchon gefunden,\\
	meine Seele zeiget ſich,\\
	trinke nun aus deinen Wunden,\\
	denn in dir verberg ich mich.
\end{description}

\cleardoublepage
\fi

\iftemplate
\includepdf[pages=-]{../out/\lypdfname.pdf}
\fi

\end{document}
