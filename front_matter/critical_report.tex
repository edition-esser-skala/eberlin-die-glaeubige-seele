\documentclass[shorttitlesize=55,tocstyle=ref-genre]{ees}

\begin{document}

\eesTitlePage

\eesCriticalReport{
  –  & –     & org  & All bass figures were added by the editor.\\
  1  & 27    & vl 2 & last \eighthNote\ in \A1: b8 \\
  7  & 19    & vla  & 2nd \eighthNote\ in \A1: \sharp f′8 \\
  10 & 49–50 & vla  & in \A1 unison with org \\
  11 & 22    & vla  & in \A1 unison with org \\
     & 55    & vl 2 & grace note missing in \A1 \\
  14 & 64    & T    & grace note missing in \A1 \\
     & 156   & T    & grace note missing in \A1 \\
}

\eesToc{
\textit{Soloists}\\[1ex]
Tochter Zion; unnamed soloist (Soprano)\\
Maria, die Mutter Jesus (Alto)\\
Evangelium (Tenore)\\
Die gläubige Seele (Tenore)

\par\bigskip
\begin{movement}{iztkomm}
  \voice[Soprano]
  Izt komm, izt komm, o rauher Sünder,\\
  kömmt, kömmt, ihr Menſchenkinder,\\
  kömmt, ſeht den Kreuzweeg an,\\
  ja ſeht was Gott vor euch gethan.\\
  Komm, Sünder! ſieh des Heilands Plagen,\\
  komm, Sünder, hilf ihm den Kreuzpflock tragen.\\
  Die Laſt, die ſeine Schulter drückt,\\
  hat deine Sünd ihm zugeſchickt.
\end{movement}

\begin{movement}{meinsohn}
  \voice[Maria, die Mutter Jesus]
  Mein Sohn! o Gottes Sohn!\\
  So geheſt du dann izt zum Tod?\\
  O harter Seelenſchmerz, o Pein! o Noth!\\
  (mir poltert faſt das Herz)\\
  Was kann wohl ich gekränkte Mutter ſagen?\\
  Ach ja! mein Sohn, den ich neun Monath lang\\
  in meinem Leibe truge,\\
  den dieſe Mutterbruſt mit ihrer Milch geſäuget –\\
  o Freud! o Troſt! o meine Luſt!\\
  du wirſt durch dieſen Kreuzespflock\\
  faſt biß zur Erd gebeuget:\\
  Man reißt dich ſchon zu deiner Pein,\\
  drum dringet ſich in meine Seele\\
  das Schwert des Schmerzens ein.\\
  So ſtärke dann, mein Gott,\\
  ach ſtärke doch den ſchwachen Geiſt,\\
  denn ſonſten deine Marter\\
  (was ich für dich gern wollt)\\
  mit dir, mein Sohn! gar ſterben heißt.
\end{movement}

\begin{movement}{weintmit}
  \voice[Maria, die Mutter Jesus]
  Weint mit mir, ihr frommen Seelen,\\
  euer Vater ſtirbt, mein Kind.\\
  Willſt, o Menſch, die Wunden zählen,\\
  ach, ſo zähle deine Sünd.\\
  Dieſe war das Maaß der Schmerzen,\\
  die ſein mater Leib erträgt,\\
  dieſe war, die ſeinem Herzen\\
  ſo viel Marter aufgelegt.
\end{movement}

\begin{movement}{sobalddie}
  \voice[Evangelium]
  Sobald die Juden nun auf Golgatha gekommen,\\
  ſo wurden ihm die Kleider weckgenommen.\\
  Drauf hat man ihn mit Wein und Gall getränkt\\
  und an das Kreuz gehenkt.

  \voice[Die gläubige Seele]
  Ach Anblick! der mein Herz zerbricht.\\
  Wie wird mein Heiland von dieſen Schärgenknechten\\
  ſo ſchändlich zugericht.\\
  Die ganze Höllenwuth iſt in der Judenbruſt beysamen.\\
  O Schrecken! ach! o Herzensgrauen!\\
  Hier reißen ihre Fingerklauen\\
  das milde Lamm mit vollem Grimm zu Boden.\\
  Sie fangen an, die Nerven und die Sehnen\\
  ihm ganz erbärmlich auszudehnen,\\
  die zarten Vaterhände\\
  mit größtem Ungeſtüm zu ſtrecken.\\
  O Jeſus! ach dein Leiden muß in mir\\
  das ſtrengſte Leid erwecken.
\end{movement}

\begin{movement}{moerderdazen}
  \voice[Die gläubige Seele]
  Mörderdazen, Mörderklauen,\\
  läßt mich nicht mehr Laſter ſchauen,\\
  denkt, was Eure Boſheit thut!\\
  Dürſtet euch nach Menſchenblut,\\
  ſo nihm meins, du Henkersbrut.\\
  Ich bin, ſo die Schuld begangen,\\
  drum läßt mich den Tod erlangen,\\
  Jeſus iſt von Sünden rein,\\
  ich, ich muß beſtrafet ſeyn.
\end{movement}

\begin{movement}{kaumals}
  \voice[Evangelium]
  Kaum als das Kreuzigen vorüber war,\\
  ſo warf die Henker Schaar\\
  das Looß um ſein Gewand,\\
  und über ſeinem Haubte ſtand\\
  der Titel ſo geſchrieben:\\
  Hier hänget Jeſus der Nazarener,\\
  der Juden König.\\
  Drauf haben alle, die vorbey gegangen,\\
  zu ſpotten angefangen,\\
  und auch die Zwey,\\
  ſo gleich zur Seite hangen,\\
  die riefen ohne Scheu:

  \voice[Chor]
  Ey ſeht den König an!\\
  Du biſt ja Gottes Sohn,\\
  ſo ſteig vom Keuz herab.\\
  Ach ſeht, wie er ihm helfen kann.

  \voice[Evangelium]
  Und eine düſtre Finſterniß entſtand\\
  hier in der ſechſten Stunde\\
  im ſelben ganzen Land.\\
  Man ſah die Sonn erblaßen\\
  biß auf die neunte Stunde.\\
  Da riefe Jeſus laut:\\
  Mein Gott, mein Gott,\\
  wie haſt du mich verlaßen?\\
  Dann ſaget er: Mich dürſtet.
\end{movement}

\begin{movement}{wiederhirsch}
  \voice[Tochter Zion]
  Wie der Hirſch ganz ſchnelle\\
  zu der Waßerquelle,\\
  die ihm ſein Schmachten heilt,\\
  mit großen/frohen Schritten eilt,\\
  ſo dürſtet dich nach meiner Seele.\\
  Doch du haſt ſie ſchon gefunden,\\
  meine Seele zeiget ſich,\\
  trinke nun aus deinen Wunden,\\
  denn in dir verberg ich mich.
\end{movement}

\begin{movement}{sodannnahm}
  \voice[Evangelium]
  Sodann nahm jäh ein Knecht ein Rohr\\
  und hielt ihm einen Schwamm mit Eßig vor,\\
  und Jeſus rief aus ganzer Kraft:\\
  Es iſt vollbracht.
\end{movement}

\begin{movement}{vollbracht}
  \voice[Die gläubige Seele]
  Es iſt vollbracht, es iſt vollbracht!\\
  Wüte nur, du Höllenmacht,\\
  deine Wuth wird izt verlacht.\\
  Höreſt du die Stimme nicht,\\
  die Gott ſelbſt am Kreuze ſpricht?\\
  Deine Kräften ſind gebrochen,\\
  meine Laſter ſind gerochen,\\
  denn das Wort, das alles macht,\\
  ſaget izt: Es iſt vollbracht.
\end{movement}

\begin{movement}{hoellemacht}
  \voice[Tochter Zion]
  Izt iſt der Hölle Macht beſinget,\\
  izt iſt der Menſchen Heil erworben.

  \voice[Evangelium]
  Und er iſt mit geneigtem Haubt geſtorben.
\end{movement}

\begin{movement}{siehoschnoeder}
  \voice[Tochter Zion]
  Sieh, o ſchnöder Sündenknecht!\\
  Sieh! wie Gott die Sünden haßet,\\
  Gott, Gott ſelber iſt erblaßet,\\
  denn dein Blut war viel zu ſchlecht.\\
  Ach, es ſpalten ſich die Steine,\\
  wein, o Sünder, Ströme weine,\\
  dieſer Tod die Felſenbruſt,\\
  nur dein Herz bewegt er nicht.
\end{movement}

\begin{movement}{otheurer}
  \voice[Die gläubige Seele]
  O theurer Sünden Zahl, o Jeſus!\\
  ach Erlöſer, du ſtirbſt,\\
  daß ich verruchtes Laſterkind\\
  doch ewig leben ſoll.\\
  O ſchmerzenreiche Stunden,\\
  die meinem Gott ſo viele Wunden\\
  und mir durch ſie das Heil gebracht.\\
  Wie kann ich wohl gekränkte Seele\\
  nach meinem Tode leben?\\
  Was kann ich für dieſe Liebe geben?\\
  Für dieſe Liebe lieb ich dich,\\
  mein Gott! mit großen Kräften.\\
  Mein Herz will ich zu dir ans Kreuze heften,\\
  mir winket ja dein halb geneigtes Haubt.\\
  Du ladeſt mich ja ſelber ein,\\
  drum ſollſt auch nach dem Tode\\
  das beſte meiner Seele ſeyn.
\end{movement}

\begin{movement}{jesumeines}
  \voice[Die gläubige Seele]
  Jeſu! meines Lebens Quelle\\
  du geliebter meiner Seele,\\
  ach, dein Haubt iſt ſchon geneigt.\\
  Die Bruſt eröffnet ſich,\\
  die Händ umfangen mich,\\
  das offne Herz nur Liebe zeigt.\\
  In deiner Seite weichen Klüfte\\
  ſchleuß ich mich ganz ſanft hinein,\\
  ſo wird mein Herz dir ſtatt der Krüfte,\\
  und deine Bruſt mein Grabmahl ſeyn.
\end{movement}

\begin{movement}{indeiner}
  \voice[Chor]
  In deiner Seite weichen Klüfte\\
  ſchleuß ich mich ganz ſanft hinein,\\
  ſo wird mein Herz dir ſtatt der Krüfte,\\
  und deine Bruſt mein Grabmahl ſeyn.
\end{movement}
}

\eesScore

\end{document}
